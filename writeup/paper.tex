\section{Introduction}

Our study originated with the boundary-integral analysis of the
Coulomb-field approximation, which has been widely used~\cite{Luo97}

Xue and Deng recently employed ellipsoidal harmonics for the
three-dielectric electrostatics problem, providing a clear exposition
and extensive analysis~\cite{Xue11}.

Onufriev and Sigalov have conducted a series of studies to improve the
Generalized Born model~\cite{Sigalov,Onufriev11}, and have established
one method based on approximating the solute-solvent boundary as an
ellipsoid~\cite{SigalovWhich}.


\section{Ellipsoidal Coordinates and Harmonics}
Following Dassios and Kariotou~\cite{Dassios03}, we define the
ellipsoid via
\begin{align}
  \frac{x^2}{\alpha_x^2} +   \frac{y^2}{\alpha_y^2} +   \frac{z^2}{\alpha_z^2} = 1,
  \end{align}
assuming without loss of generality that $\alpha_x > \alpha_y >
\alpha_z$.  We also define the semifocal distances
\begin{align}
  h_x &= \sqrt{\alpha_y^2-\alpha_z^2}\\
  h_y &= \sqrt{\alpha_x^2-\alpha_z^2}\\
  h_z &= \sqrt{\alpha_x^2-\alpha_y^2}.
\end{align}
We will also define $h_z = h$ and $h_y = k$.
The ellipsoidal coordinate system $(\rho, \mu, \nu)$ and the Cartesian
coordinates $(x, y, z)$ are related by
\begin{align}\label{eq:cartesianSq}
  x^2 &= \frac{\rho^2 \mu^2 \nu^2}{h^2_y h^2_z} \\
  y^2 &= \frac{(\rho^2 - h^2_z)(\mu^2 - h^2_z)(h^2_z - \nu^2)}{h^2_x h^2_z} \\
  z^2 &= \frac{(\rho^2 - h^2_y)(h^2_y - \mu^2)(h^2_y - \nu^2)}{h^2_x h^2_y}.
\end{align}
There is a sign ambiguity here, which can be resolved by appealing to the Lam\'e functions of first order. These are
just the dipoles, and as such have the same sign as the traditional dipoles. Therefore, using $s_x = \mathrm{sgn}(x)$
\begin{align}
  s_x &= \mathrm{sgn}\left(H^0_1(\rho) H^0_1(\mu) H^0_1(\nu)\right) \\
  s_y &= \mathrm{sgn}\left(H^1_1(\rho) H^1_1(\mu) H^1_1(\nu)\right) \\
  s_z &= \mathrm{sgn}\left(H^2_1(\rho) H^2_1(\mu) H^2_1(\nu)\right),
\end{align}
however this just moves the sign ambiguity to the Lam\'e functions. We have, from Romain Table III or by taking square
roots of Eq.~\ref{eq:cartesianSq},
\begin{align}
  h_y h_z x &= \rho \mu \nu \\
  h_x h_z y &= \sqrt{\rho^2 - h^2_z} \sqrt{\mu^2 - h^2_z} \sqrt{h^2_z - \nu^2} \\
  h_x h_y z &= \sqrt{\rho^2 - h^2_y} \sqrt{h^2_y - \mu^2} \sqrt{h^2_y - \nu^2},
\end{align}
where the sign of the square roots must be chosen consistently. We choose the sign of each square root as the product of
the sign of argument and a sign determined by the semifocal axis. Thus we have
\begin{align}\label{eq:sign}
  s_x &= s_\rho s_\mu s_\nu \\
  s_y &= (s_\rho s_{h_z}) (s_\mu s_{h_z}) (s_\nu s_{h_z}) = s_\rho s_\mu s_\nu s_{h_z} \\
  s_z &= (s_\rho s_{h_y}) (s_\mu s_{h_y}) (s_\nu s_{h_y}) = s_\rho s_\mu s_\nu s_{h_y}.
\end{align}
Multiplying the first and second, and first and third equations,
\begin{align}
  s_x s_y &= s_{h_z} \\
  s_x s_z &= s_{h_y}.
\end{align}
Now our system has the solution
\begin{align}
  s_\rho &= s_x s_y s_z \\
  s_\mu  &= s_x s_y \\
  s_\nu  &= s_x s_z,
\end{align}
which we can check using Eq.~\ref{eq:sign},
\begin{align}\label{eq:sign}
  s_x &= (s_x s_y s_z) (s_x s_y) (s_x s_z) \\
  s_y &= s_x (s_x s_y) \\
  s_z &= s_x (s_x s_z).
\end{align}
We use the sign assignment
\begin{align}
  s_\mu &= s_{h_z} \\
  s_\nu &= s_{h_y}.
\end{align}
when computing the sign of $\psi^p_n$ used in the Lam\'e functions, and found in Romain Table II.

Loosely speaking, the ellipsoidal coordinate $\rho$ is analogous to
the radius in spherical coordinates, and $(\mu,\nu)$ are analogous to
$(\theta, \phi)$, with the ellipsoid surface defined by $\rho =
\alpha_1$, with $h_y^2 < \mu^2 < h_z^2$ and $0 < \nu^2 < h_z^2$.  We
caution the reader that the analogy is more conceptual than practical.
As a simple example, as the ellipsoid approaches a sphere
($\alpha_x=\alpha_y=\alpha_z$) or spheroid ($\alpha_x=\alpha_y$ for a
prolate spheroid and $\alpha_y=\alpha_z$ for an oblate spheroid), the
semifocal distances approach zero; these limits necessitate more
complicated analysis~\cite{Hobson31,Ritter,Dassios03}.  Hobson's text
presents a detailed derivation of the coordinate system, using
different notation~\cite{Hobson31}.

The Laplace equation is separable in this coordinate system,
remarkably each coordinate satisfies the Lame differential
equation
\begin{align}
  \end{align}
for $h_y < \rho < +\infty$ and the above ranges for $\mu$ and
$\nu$~\cite{Darwin,Nevin,Hobson31,Dassios03}.  As for spherical
harmonics, there exist $2n+1$ solutions for each degree $n$; defining
$r=\frac{1}{2}n$ if $n$ is even and $r=\frac{1}{2}(n+1)$ if $n$ is
odd, there are $r+1$ solutions labeled $K$, $n-r$ solutions labeled
$L$, $n-r$ labeled $M$, and $r$ labeled $N$.  

Unlike
spherical harmonics, however, the ellipsoidal harmonics cannot be
determined analytically beyond $n=3$, though they can be computed
numerically~\cite{Ritter}.  Here we address only low-order surface
ellipsoidal harmonics ($n=0,1,2$),
\begin{align}
H_0^1(\mu,\nu)&=K_0^1(\mu)K_0^1(\nu)&=1\\
H_1^1(\mu,\nu)&=K_1^1(\mu)K_1^1(\nu)&=x\\
H_1^2(\mu,\nu)&=L_1^1(\mu)L_1^1(\nu)&=y\\
H_1^3(\mu,\nu)&=M_1^1(\mu)M_1^1(\nu)&=z
  \end{align}

Ritter showed that the ellipsoidal harmonics $H_n^m(\mu,\nu)$ are
eigenfunctions of $\mathcal{K}$, with eigenvalues
\begin{align}
  \lambda_n^m =-1 + \frac{2 \alpha_x \alpha_y \alpha_z}{2 n + 1}K_n^m(\alpha_1)\left.\left(\frac{\partial}{\partial \rho}L_n^m(\rho)\right)\right|_{\alpha_1}.
  \end{align}

\section{Estimation of the Protein Ellipsoid}
Sigalov \textit{et al.} have presented a method to estimate the shape
of a solute molecule as an ellipsoid~\cite{Sigalov06}, which we use in
this work as a simple and validated approach.  Briefly, the $N$ atoms
are modeled as hard spheres, and the mass of the $i$th atom is defined
as $m_i = a_i^3$ where $a_i$ is its radius.  The molecule is
translated so its center of mass is the origin, and then rotated to
diagonalize the inertia tensor defined by
\begin{align}
  \end{align}

The ellipsoid semiaxes are then given by
\begin{align}
  \alpha_x &= \sqrt{\frac{5}{2M}(-I_{xx}+I_{yy}+I_{zz})}\\
  \alpha_y &= \sqrt{\frac{5}{2M}(+I_{xx}-I_{yy}+I_{zz})}\\
  \alpha_z &= \sqrt{\frac{5}{2M}(+I_{xx}+I_{yy}-I_{zz})}
  \end{align}

\section{Reduced Spectrum Approach}
We start with the sphere problem, because the electric field operator
$K$ is symmetric in this special case, so consequently its
eigenvectors are orthogonal.  In other words, the eigendecomposition
of $K = V \Lambda V^{-1}$ is also written $K = V \Lambda V^T$ where
the $i$th column of $V$ is the $i$th eigenvector, associated with
$\Lambda_{ii}= \lambda_i$ and $\Lambda$ is diagonal.  The exact
solution to
\begin{align}
  \left(I + \hat{\epsilon}K\right) \sigma = x
  \end{align}
is
\begin{align}
\sigma = \sum_{i=1} \left(1+\hat{\epsilon}\lambda_i\right) v_i v_i^T x.
  \end{align}
The BIBEE/M and BIBEE/I approximations start from the trivial
decomposition
\begin{align}
  x = v_1 v_1^T x + \left(I - v_1 v_1^T \right) x
  \end{align}
where $v_1$ is the (normalized) constant vector.  Then
\begin{align}
  (I -\hat{\epsilon}K) \sigma = x
\end{align}
can be approximated as
\begin{align}
  \hat{\sigma} = \left(1+\hat{\epsilon}(-\frac{1}{2})\right)^{-1} v_1 v_1^T x
 + \left(1+\hat{\epsilon}\lambda^*\right)^{-1}\left(I- v_1 v_1^T \right) x
  \end{align}
where $\lambda^* = 0$ in BIBEE/M and takes a fitted value in BIBEE/I
(see Figure 1).  The more general expression
\begin{align}
  x = v_1 v_1^T x + v_2 v_2^T x + \ldots + v_k v_k^T x + \left(I-
  \begin{sbmatrix}{ccc}v_1 &\cdots&v_k\end{sbmatrix}
  \begin{sbmatrix}{c}v_1^T\\\vdots\\v_k^T\end{sbmatrix}
  \right)x
  \end{align}
leads to the approximate solution
\begin{align}
  \hat{\sigma} = & \sum_{i=1}^k \left(1+\hat{\epsilon}\lambda_i\right)^{-1} v_i v_i^T x \\
  & + \left(1+\hat{\epsilon}\lambda^*\right)^{-1}\left(I-
  \begin{sbmatrix}{ccc}v_1 &\cdots&v_k\end{sbmatrix}
  \begin{sbmatrix}{c}v_1^T\\\vdots\\v_k^T\end{sbmatrix}
  \right)x.
  \end{align}
For general surfaces $K$ is not symmetric, so it is not known whether
the eigenvector matrix $V$ is orthogonal.  However, by writing
\begin{align}
K=  \begin{sbmatrix}{ccc}| & & |\\v_1 &\cdots&v_k\\ | & & | \end{sbmatrix}
\begin{sbmatrix}{ccc}\lambda_1 & & \\ & \ddots & \\ & & \lambda_k \end{sbmatrix}
  \begin{sbmatrix}{ccc} - & \tilde{v}_1 & -\\&\vdots&\\-&\tilde{v}_k&-\end{sbmatrix}
\end{align}
where $\tilde{v}_i$ denotes the $i$th row of $V^{-1}$, we may still write
the exact solution as
\begin{align}
  \sigma = \sum \left(1+\hat{\epsilon}\lambda_i\right)^{-1} v_i \tilde{v}_i x
  \end{align}
The central approximation in this work is that we assume the
eigenvectors are orthogonal with respect to the weighting function
associated with the ellipsoidal harmonics~\cite{Dassios}.

\section{Measuring the Effect of Eigenvector Approximation}

If the molecular surface were actually ellipsoidal, then the
projection approximation would be exact for the monopole and dipole
modes of the integral operator modes.  However, the 
