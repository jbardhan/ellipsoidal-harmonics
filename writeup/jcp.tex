\documentclass[preprint, letterpaper, nobibnotes, aps, superscriptaddress,endfloats*,prb]{revtex4-1}

\usepackage{graphicx}
\usepackage{amssymb}
\usepackage{amsmath}
\usepackage{color}
\usepackage{url}
\usepackage{graphicx}        % standard LaTeX graphics tool
                             % when including figure files
%\usepackage{minted}

% Comments package


\def\vf{{\bf f}}
\def\vrp{{\bf r}'}
\def\vr{{\bf r}}
\def\vu{{\bf u}}
\def\vv{{\bf v}}
\def\vx{{\bf x}}
\def\vn{{\bf n}}
\def\code#1{{\tt #1}}
\def\enum#1{{\sc #1}}
\def\class#1{{\tt\bf #1}}
\def\function#1{\code{#1}}

% Color macros
\newcommand{\blue}{\textcolor{blue}}
\newcommand{\green}{\textcolor{green}}
\newcommand{\red}{\textcolor{red}}
\newcommand{\brown}{\textcolor{brown}}
\newcommand{\cyan}{\textcolor{cyan}}
\newcommand{\magenta}{\textcolor{magenta}}
\newcommand{\yellow}{\textcolor{yellow}}

% Author notes
%\usepackage[colorinlistoftodos, textwidth=4cm, shadow]{todonotes}
%\newcommand{\note}[1]{\todo[color=orange!40,inline]{{\bf Note:} #1}}
%\newcommand{\jay}[1]{\todo[color=blue!40,inline]{{\bf Jay:} #1}}
%\newcommand{\matt}[1]{\todo[color=green!40,inline]{{\bf Matt:} #1}}
%\newcommand{\fixme}[1]{\todo[color=red!40,inline]{{\bf FIXME:} #1}}
% \newcommand{\note}[1]{}
% \newcommand{\andy}[1]{}
% \newcommand{\matt}[1]{}
% \newcommand{\fixme}[1]{}
\let\comment = \note

\newcommand{\arxiv}{\let\arxivFig\matt}

\def\Xint#1{\mathchoice
   {\XXint\displaystyle\textstyle{#1}}%
   {\XXint\textstyle\scriptstyle{#1}}%
   {\XXint\scriptstyle\scriptscriptstyle{#1}}%
   {\XXint\scriptscriptstyle\scriptscriptstyle{#1}}%
   \!\int}
\def\XXint#1#2#3{{\setbox0=\hbox{$#1{#2#3}{\int}$}
     \vcenter{\hbox{$#2#3$}}\kern-.5\wd0}}
\def\ddashint{\Xint=}
\def\dashint{\Xint-}
\newenvironment{sbmatrix}[1]{\left[ \begin{array}{#1}}%
	{\end{array} \right]}


\begin{document}

\title{Mathematical Analysis of the BIBEE Approximation for Molecular Solvation: Exact Results for Spherical Inclusions}
\author{Jaydeep P. Bardhan}
\affiliation{Dept. of Molecular Biophysics and Physiology, Rush University Medical Center, Chicago IL 60612}
\author{Matthew G. Knepley}
\affiliation{Computation Institute, The University of Chicago, Chicago IL 60637}
\input{abstract}

\maketitle

%\thanks{Put in grant numbers}

\section{Introduction}

Our study originated with the boundary-integral analysis of the
Coulomb-field approximation, which has been widely used~\cite{Luo97}

Xue and Deng recently employed ellipsoidal harmonics for the
three-dielectric electrostatics problem, providing a clear exposition
and extensive analysis~\cite{Xue11}.

Onufriev and Sigalov have conducted a series of studies to improve the
Generalized Born model~\cite{Sigalov,Onufriev11}, and have established
one method based on approximating the solute-solvent boundary as an
ellipsoid~\cite{SigalovWhich}.


\section{Ellipsoidal Coordinates and Harmonics}
Following Dassios and Kariotou~\cite{Dassios03}, we define the
ellipsoid via
\begin{align}
  \frac{x^2}{\alpha_x^2} +   \frac{y^2}{\alpha_y^2} +   \frac{z^2}{\alpha_z^2} = 1,
  \end{align}
assuming without loss of generality that $\alpha_x > \alpha_y >
\alpha_z$.  We also define the semifocal distances
\begin{align}
  h_x &= \sqrt{\alpha_y^2-\alpha_z^2}\\
  h_y &= \sqrt{\alpha_x^2-\alpha_z^2}\\
  h_z &= \sqrt{\alpha_x^2-\alpha_y^2}.
\end{align}
We will also define $h_z = h$ and $h_y = k$.
The ellipsoidal coordinate system $(\rho, \mu, \nu)$ and the Cartesian
coordinates $(x, y, z)$ are related by
\begin{align}\label{eq:cartesianSq}
  x^2 &= \frac{\rho^2 \mu^2 \nu^2}{h^2_y h^2_z} \\
  y^2 &= \frac{(\rho^2 - h^2_z)(\mu^2 - h^2_z)(h^2_z - \nu^2)}{h^2_x h^2_z} \\
  z^2 &= \frac{(\rho^2 - h^2_y)(h^2_y - \mu^2)(h^2_y - \nu^2)}{h^2_x h^2_y}.
\end{align}
There is a sign ambiguity here, which can be resolved by appealing to the Lam\'e functions of first order. These are
just the dipoles, and as such have the same sign as the traditional dipoles. Therefore, using $s_x = \mathrm{sgn}(x)$
\begin{align}
  s_x &= \mathrm{sgn}\left(H^0_1(\rho) H^0_1(\mu) H^0_1(\nu)\right) \\
  s_y &= \mathrm{sgn}\left(H^1_1(\rho) H^1_1(\mu) H^1_1(\nu)\right) \\
  s_z &= \mathrm{sgn}\left(H^2_1(\rho) H^2_1(\mu) H^2_1(\nu)\right),
\end{align}
however this just moves the sign ambiguity to the Lam\'e functions. We have, from Romain Table III or by taking square
roots of Eq.~\ref{eq:cartesianSq},
\begin{align}
  h_y h_z x &= \rho \mu \nu \\
  h_x h_z y &= \sqrt{\rho^2 - h^2_z} \sqrt{\mu^2 - h^2_z} \sqrt{h^2_z - \nu^2} \\
  h_x h_y z &= \sqrt{\rho^2 - h^2_y} \sqrt{h^2_y - \mu^2} \sqrt{h^2_y - \nu^2},
\end{align}
where the sign of the square roots must be chosen consistently. We choose the sign of each square root as the product of
the sign of argument and a sign determined by the semifocal axis. Thus we have
\begin{align}\label{eq:sign}
  s_x &= s_\rho s_\mu s_\nu \\
  s_y &= (s_\rho s_{h_z}) (s_\mu s_{h_z}) (s_\nu s_{h_z}) = s_\rho s_\mu s_\nu s_{h_z} \\
  s_z &= (s_\rho s_{h_y}) (s_\mu s_{h_y}) (s_\nu s_{h_y}) = s_\rho s_\mu s_\nu s_{h_y}.
\end{align}
Multiplying the first and second, and first and third equations,
\begin{align}
  s_x s_y &= s_{h_z} \\
  s_x s_z &= s_{h_y}.
\end{align}
Now our system has the solution
\begin{align}
  s_\rho &= s_x s_y s_z \\
  s_\mu  &= s_x s_y \\
  s_\nu  &= s_x s_z,
\end{align}
which we can check using Eq.~\ref{eq:sign},
\begin{align}\label{eq:sign}
  s_x &= (s_x s_y s_z) (s_x s_y) (s_x s_z) \\
  s_y &= s_x (s_x s_y) \\
  s_z &= s_x (s_x s_z).
\end{align}
We use the sign assignment
\begin{align}
  s_\mu &= s_{h_z} \\
  s_\nu &= s_{h_y}.
\end{align}
when computing the sign of $\psi^p_n$ used in the Lam\'e functions, and found in Romain Table II.

Loosely speaking, the ellipsoidal coordinate $\rho$ is analogous to
the radius in spherical coordinates, and $(\mu,\nu)$ are analogous to
$(\theta, \phi)$, with the ellipsoid surface defined by $\rho =
\alpha_1$, with $h_y^2 < \mu^2 < h_z^2$ and $0 < \nu^2 < h_z^2$.  We
caution the reader that the analogy is more conceptual than practical.
As a simple example, as the ellipsoid approaches a sphere
($\alpha_x=\alpha_y=\alpha_z$) or spheroid ($\alpha_x=\alpha_y$ for a
prolate spheroid and $\alpha_y=\alpha_z$ for an oblate spheroid), the
semifocal distances approach zero; these limits necessitate more
complicated analysis~\cite{Hobson31,Ritter,Dassios03}.  Hobson's text
presents a detailed derivation of the coordinate system, using
different notation~\cite{Hobson31}.

The Laplace equation is separable in this coordinate system,
remarkably each coordinate satisfies the Lame differential
equation
\begin{align}
  \end{align}
for $h_y < \rho < +\infty$ and the above ranges for $\mu$ and
$\nu$~\cite{Darwin,Nevin,Hobson31,Dassios03}.  As for spherical
harmonics, there exist $2n+1$ solutions for each degree $n$; defining
$r=\frac{1}{2}n$ if $n$ is even and $r=\frac{1}{2}(n+1)$ if $n$ is
odd, there are $r+1$ solutions labeled $K$, $n-r$ solutions labeled
$L$, $n-r$ labeled $M$, and $r$ labeled $N$.  

Unlike
spherical harmonics, however, the ellipsoidal harmonics cannot be
determined analytically beyond $n=3$, though they can be computed
numerically~\cite{Ritter}.  Here we address only low-order surface
ellipsoidal harmonics ($n=0,1,2$),
\begin{align}
H_0^1(\mu,\nu)&=K_0^1(\mu)K_0^1(\nu)&=1\\
H_1^1(\mu,\nu)&=K_1^1(\mu)K_1^1(\nu)&=x\\
H_1^2(\mu,\nu)&=L_1^1(\mu)L_1^1(\nu)&=y\\
H_1^3(\mu,\nu)&=M_1^1(\mu)M_1^1(\nu)&=z
  \end{align}

Ritter showed that the ellipsoidal harmonics $H_n^m(\mu,\nu)$ are
eigenfunctions of $\mathcal{K}$, with eigenvalues
\begin{align}
  \lambda_n^m =-1 + \frac{2 \alpha_x \alpha_y \alpha_z}{2 n + 1}K_n^m(\alpha_1)\left.\left(\frac{\partial}{\partial \rho}L_n^m(\rho)\right)\right|_{\alpha_1}.
  \end{align}

\section{Estimation of the Protein Ellipsoid}
Sigalov \textit{et al.} have presented a method to estimate the shape
of a solute molecule as an ellipsoid~\cite{Sigalov06}, which we use in
this work as a simple and validated approach.  Briefly, the $N$ atoms
are modeled as hard spheres, and the mass of the $i$th atom is defined
as $m_i = a_i^3$ where $a_i$ is its radius.  The molecule is
translated so its center of mass is the origin, and then rotated to
diagonalize the inertia tensor defined by
\begin{align}
  \end{align}

The ellipsoid semiaxes are then given by
\begin{align}
  \alpha_x &= \sqrt{\frac{5}{2M}(-I_{xx}+I_{yy}+I_{zz})}\\
  \alpha_y &= \sqrt{\frac{5}{2M}(+I_{xx}-I_{yy}+I_{zz})}\\
  \alpha_z &= \sqrt{\frac{5}{2M}(+I_{xx}+I_{yy}-I_{zz})}
  \end{align}

\section{Reduced Spectrum Approach}
We start with the sphere problem, because the electric field operator
$K$ is symmetric in this special case, so consequently its
eigenvectors are orthogonal.  In other words, the eigendecomposition
of $K = V \Lambda V^{-1}$ is also written $K = V \Lambda V^T$ where
the $i$th column of $V$ is the $i$th eigenvector, associated with
$\Lambda_{ii}= \lambda_i$ and $\Lambda$ is diagonal.  The exact
solution to
\begin{align}
  \left(I + \hat{\epsilon}K\right) \sigma = x
  \end{align}
is
\begin{align}
\sigma = \sum_{i=1} \left(1+\hat{\epsilon}\lambda_i\right) v_i v_i^T x.
  \end{align}
The BIBEE/M and BIBEE/I approximations start from the trivial
decomposition
\begin{align}
  x = v_1 v_1^T x + \left(I - v_1 v_1^T \right) x
  \end{align}
where $v_1$ is the (normalized) constant vector.  Then
\begin{align}
  (I -\hat{\epsilon}K) \sigma = x
\end{align}
can be approximated as
\begin{align}
  \hat{\sigma} = \left(1+\hat{\epsilon}(-\frac{1}{2})\right)^{-1} v_1 v_1^T x
 + \left(1+\hat{\epsilon}\lambda^*\right)^{-1}\left(I- v_1 v_1^T \right) x
  \end{align}
where $\lambda^* = 0$ in BIBEE/M and takes a fitted value in BIBEE/I
(see Figure 1).  The more general expression
\begin{align}
  x = v_1 v_1^T x + v_2 v_2^T x + \ldots + v_k v_k^T x + \left(I-
  \begin{sbmatrix}{ccc}v_1 &\cdots&v_k\end{sbmatrix}
  \begin{sbmatrix}{c}v_1^T\\\vdots\\v_k^T\end{sbmatrix}
  \right)x
  \end{align}
leads to the approximate solution
\begin{align}
  \hat{\sigma} = & \sum_{i=1}^k \left(1+\hat{\epsilon}\lambda_i\right)^{-1} v_i v_i^T x \\
  & + \left(1+\hat{\epsilon}\lambda^*\right)^{-1}\left(I-
  \begin{sbmatrix}{ccc}v_1 &\cdots&v_k\end{sbmatrix}
  \begin{sbmatrix}{c}v_1^T\\\vdots\\v_k^T\end{sbmatrix}
  \right)x.
  \end{align}
For general surfaces $K$ is not symmetric, so it is not known whether
the eigenvector matrix $V$ is orthogonal.  However, by writing
\begin{align}
K=  \begin{sbmatrix}{ccc}| & & |\\v_1 &\cdots&v_k\\ | & & | \end{sbmatrix}
\begin{sbmatrix}{ccc}\lambda_1 & & \\ & \ddots & \\ & & \lambda_k \end{sbmatrix}
  \begin{sbmatrix}{ccc} - & \tilde{v}_1 & -\\&\vdots&\\-&\tilde{v}_k&-\end{sbmatrix}
\end{align}
where $\tilde{v}_i$ denotes the $i$th row of $V^{-1}$, we may still write
the exact solution as
\begin{align}
  \sigma = \sum \left(1+\hat{\epsilon}\lambda_i\right)^{-1} v_i \tilde{v}_i x
  \end{align}
The central approximation in this work is that we assume the
eigenvectors are orthogonal with respect to the weighting function
associated with the ellipsoidal harmonics~\cite{Dassios}.

\section{Measuring the Effect of Eigenvector Approximation}

If the molecular surface were actually ellipsoidal, then the
projection approximation would be exact for the monopole and dipole
modes of the integral operator modes.  However, the 


\bibliographystyle{unsrt}
\bibliography{paper}

\end{document}

The spherical harmonic analysis suggested a simple ansatz that
possessed significantly better accuracy than either of the BIBEE
models, but could not be resolved as a simple modified boundary
condition as can the others.  As a result, the simple ansatz cannot
readily be implemented outside the sphere problem, which is
discouraging.  This result seemed at first to contradict the
separability condition, but closer inspection reveals that it odes
not, for the following reason.  The designed ansatz relates the
fixed-charge Coulomb potential to the reaction-field potential at the
surface; however, the BIBEE separation of variables occurs in the
determination of the approximate surface charge distribution.  The
estimation of electrostatic free energies using separable
approximations has been highlighted very recently as a way to
incorporate dielectric effects into classical density functional
theory (DFT)~\cite{Gillespie}; thus, the separability approach to
analyzing electrostatic models may have broader applications than the
one described here.


\section{Comparison to the GB$\epsilon$ Model}\label{sec:GBeps}


\subsection{The Electrostatic Radius As A Spherical Capacitance Problem}\label{sec:capacitance}
The GB$\epsilon$ model computes the ``electrostatic radius'' of a
molecule by employing the Born expression for the self-energy of one
point charge in the molecule as $\beta \to \infty$.  In introducing
the method, Sigalov \textit{et al.} discuss several aspects of the
physical significance of this parameter, including the fact that all
of the charges' self-energies approach the same value in this
limit~\cite{Sigalov05}.  Although the authors do note that this limit
corresponds to the solute becoming a conductor, apparently one
relationship that has not been discussed is that the electrostatic
radius may be interpreted as an \textit{equivalent spherical
  capacitor}.  Consequently, methods for calculating capacitance may
be valuable for GB$\epsilon$ understanding and implementation.

A linear capacitor is defined by the relation $Q = C V$, where $Q$ is
the net charge on the conductor surface, $C$ is the capacitance, and
$V$ is the potential of the conductor relative to ground; the energy
stored in such a configuration is $E = \frac{1}{2} Q^2/C$.  Because
the electric field must be zero inside a conductor, the potential
everywhere inside must be constant, including at the boundary
$\Omega$.  Furthermore, as Sigalov \textit{et al.}  discussed, the
total amount of surface charge is exactly equal in magnitude and
opposite in sign to the net charge $\int_{V_1} \rho(\vr) dV$.  We may
therefore use the following scheme to calculate the self-energy of a
+1$e$ point charge, where $e$ is the electron charge $1.6\times
10^{-19}$ Coulombs.  First, one assumes that the molecule has no
charges inside it, so that the Laplace equation holds in both $V_1$
and $V_2$. second, set the potential at $\Omega$ to be $1$ Volt and
compute the equivalent capacitance $C$ of the molecule.  Finally, one
calculates $\frac{1}{2} e^2/C$.  It is well-known that the capacitance
problem can be solved using the first-kind Fredholm boundary integral
equation
\begin{equation}
  \varphi(\vr) = \int_\Omega \sigma(\vrp) \cdot \frac{1}{4\pi ||\vr-\vrp||} d^2\vrp 
\end{equation}
where $\varphi(\vr)$ is the potential at the conductor surface and again
$\sigma(\vr)$ is the induced charge density.  

We have demonstrated this relationship using the residue glutamate
with neutral capping groups, built using the software
VMD~\cite{Humphrey96} and the CHARMM force field~\cite{MacKerell98}.
Plotted in Figure~\ref{fig:radius-as-capacitance} are the relative RMS
deviations of the point charges' self-energies compared to the energy
stored in the ``capacitor'' molecule when a total of $-1 e$ charge
is situated on its surface.
\begin{figure}[ht!]
\centering
\resizebox{3.0in}{!}{\includegraphics{figures/glutamate-capacitance.eps}}
\caption{Illustration of the relationship between the capacitance
  problem and a solute's electrostatic radius in the GB$\epsilon$
  model, using the residue glutamate with neutral capping groups.
  Plotted as a function of the dielectric ratio
  $\beta=\epsilon_1/\epsilon_2$ are the relative deviations between
  the ``perfect'' self-energies for $+1e$ charges (calculated using
  boundary-element methods) and the energy stored in a capacitor of
  the same shape when the total surface charge is equal to $-1 e$.  In
  these calculations, $\epsilon_2 = 80$ always and $\epsilon_1$ varies
  from 2 to $10^5$.}\protect\label{fig:radius-as-capacitance}
\end{figure}

\subsection{Dependence of BIBEE Accuracy on Dielectric Constants}
% analytical results: BIBEE sucks when charges are in high dielectric
% numerical results: 

\subsection{Effective Dielectric Constants}
% overscreening: +/- surface charges, vary the theta between them
%    exact; GB/CFA; GBeps; BIBEE
% sphere with spherical cavity: note that 

\subsection{Preservation of Reaction-Potential Operator Eigenvectors}
% why are eigenvectors important
% we have shown that BIBEE eigenvectors are exact for the sphere
% how are GBeps eigenvectors for sphere?
% how are GBeps eigenvectors for peptide (choose antimicrobial?)

\section{Discussion}
% effective dielectric constants
\begin{enumerate}
\item Overscreening (dangerous word because it shows up elsewhere
  meaning other things, we propose the alternative X)
  \end{enumerate}

% Capacitance
\begin{enumerate}
\item The problem of rapidly estimating capacitances has received
  extensive treatment due to its practical importance in the design
  of integrated circuits~\cite{foo}.
\item Fast methods for solving these integral equations are among the
  first applications of fast BEM methods
\item Because the convergence of an iterative method such as GMRES is
  rapid for these problems, one may be able to efficiently maintain a
  highly accurate estimate of the molecule's e-s radius (show that
  it's important with an example), i.e. not fully solving the
  capacitance problem but estimating it by a few iterations ie low
  tolerance solve.
\item Tree codes and FMM may all be used, cite Onufriev other paper,
 since implementations suitable already exist, these should be
readily integrated with no heuristics necessary.
\item Point out Sigalov06 problems with capacitance estimation heuristics
  that they introduced, but do so nicely.
\item We have not run the finite-difference calculations reported by
  Sigalov05 but note that the challenges of high accuracy capacitance
  calculations motivated the development of boundary-integral methods
  for their computation (old papers, ask Jacob).  The large errors
  reported in the GBeps paper may result from the same errors--it
  becomes essentially an exterior problem.
\end{enumerate}

% dielectric constant dependence
\begin{enumerate}
\item We have found that the BIBEE methods do not exhibit nearly the
  accuracy that GB$\epsilon$ does while varying the dielectric
  constants.  
\item This result was anticipated given two facts.  First, the
  polarization free energy changes from being negative in sign
  (i.e. favorable transfer out of a uniform medium of dielectric
  constant $\epsilon_1$ and into one with $\epsilon_2$) to a positive
  quantity if $\epsilon_1 > \epsilon_2$.  Second, BIBEE/P and
  BIBEE/CFA make complementary assumptions about the important
  eigenmodes and one keeps the constant field, one keeps the highly
  varying fields.  The importance of these two components switches
  place as $\beta$ goes from zero to infinity (check).
\item Eigenvectors are still well preserved, which is in keeping with
  our earlier analysis (cite paper) for a sphere that the BIBEE
  eigenvectors are exact; that analysis did not rest on the particular
  dielectric constants.  However, the eigenvectors are also very good
  even when the estimated energies are highly inaccurate.
\item Noteworthy that the eigenvectors of BIBEE reaction-potential
  operator are invariant with respect to dielectric constant (check):
  are the eigenvectors of the actual problem?  more analysis is
  needed.
\item We are working to develop a dielectric-dependent BIBEE given our
  emphasis on general approximation schemes, and motivated by the
  success of the GB$\epsilon$ model.  However, for protein problems
  this should not be a big deal.
\end{enumerate}

% general
\begin{enumerate}
\item We are approximating the induced surface charge, whereas the
  basis for GB is the approximation of the reaction field potential
  directly.
\item The latter might be argued to be the easier approximation to
  make, since it is smoother (it is the Coulomb convolution integral
  over the induced charge, one degree smoother mathematically)
\item We have made some progress but clearly there is still a long way
  to go.  BIBEE is not currently as accurate as modern GB approaches.
\item The mathematical basis for the BIBEE approach is strong,
  however, and we are continuing to explore how the extensive
  analytical efforts on GB may be leveraged in a mathematically
  general approximate electrostatic model.
\item Our strategy also benefits from a century of mathematical
  analysis of the boundary integral operator that BIBEE
  approximates~\cite{Plemelj11,Ahner86,Ahner94,Ahner94_2,Martensen99}.
\item Specifically, it is known that the sum of its eigenvalues are an
  invariant for a sphere, spheroid, or ellipsoid, a property that may
  well explain the remarkable accuracy of the $\alpha$ parameter in
  the GB$\epsilon$ model for a wide range of realistic problems, when
  it has only been parameterized for the sphere.
\end{enumerate}

We also obtain an explanation for BIBEE's excellent preservation of
the reaction-potential matrix eigenvectors: for the spherical cavity,
the BIBEE approximation exactly preserves the eigenfunctions of the
actual reaction-potential operator.  Further analysis will be required
to determine whether this rigorous result holds also for nontrivial
geometries, but the present finding at least provides a theoretical
justification for the remarkable behavior of the BIBEE approximation,
and earlier results on peptides (i.e. nonspherical solutes) illustrate
its quality for problems that do not yet have as strong a theoretical
understanding~\cite{Bardhan08_BIBEE}.  Finally, our analysis
illustrates the BIBEE models can be interpreted as Poisson problems
with modified boundary conditions, and offers a simple mathematical
model to explain why the BIBEE/CFA and BIBEE/P variants exhibit
complementary accuracy for charge distributions that generate smoothly
or rapidly varying normal electric fields.


The BIBEE algorithm should be contrasted with the surface-charge
generalized Born (SCGB) model of Fedichev et
al.~\cite{Fedichev09,Fedichev09_2}.  In that approach, one finds
effective Born radii using either the CFA or other methods, and then
similar to BIBEE computes an approximate induced surface charge by
assuming that the surface charge density at a point ${\bf r}$ on the
surface is proportional to the normal electric field there.  However,
unlike BIBEE, the authors scale each point charge's contribution to
the surface charge density by the charge's Born radius.  A more
detailed analysis of this approach is outside the scope of this work;
however, as the authors note, their approach generates the exact
solution for a sphere.


% general
\begin{enumerate}
\item We are approximating the induced surface charge, whereas the
  basis for GB is the approximation of the reaction field potential
  directly.
\item The latter might be argued to be the easier approximation to
  make, since it is smoother (it is the Coulomb convolution integral
  over the induced charge, one degree smoother mathematically)
\item We have made some progress but clearly there is still a long way
  to go.  BIBEE is not currently as accurate as modern GB approaches.
\item The mathematical basis for the BIBEE approach is strong,
  however, and we are continuing to explore how the extensive
  analytical efforts on GB may be leveraged in a mathematically
  general approximate electrostatic model.
\item Our strategy also benefits from a century of mathematical
  analysis of the boundary integral operator that BIBEE
  approximates~\cite{Plemelj11,Ahner86,Ahner94,Ahner94_2,Martensen99}.
\item Specifically, it is known that the sum of its eigenvalues are an
  invariant for a sphere, spheroid, or ellipsoid, a property that may
  well explain the remarkable accuracy of the $\alpha$ parameter in
  the GB$\epsilon$ model for a wide range of realistic problems, when
  it has only been parameterized for the sphere.
\end{enumerate}
